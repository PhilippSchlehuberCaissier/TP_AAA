\documentclass[tikz,svgnames]{standalone}
\usetikzlibrary{arrows.meta,shadows,positioning}
\usepackage[most]{tcolorbox}
\usepackage{adjustbox}

\begin{document}

\tikzset{
  a/.style={minimum height=2cm, align=left, right, font=\tt\small, xshift=5pt},
  cell00/.style={},
  cell10/.style={},
  part1/.pic={
    \draw[-latex] (0,0) -- ++(6,0) node[right]{\itshape Lifetime};
 
 

    \path (1.5,0) node {$\mid$} node[below] {Global};
    \path (4.5,0) node {$\mid$} node[below] {Local};


    \filldraw (0,0) circle(3pt);


    \node[a, cell10] at (3,1) {
      void foo() \{\\
      ~~int {\color{DarkGreen} i} = 0;\\
      \}
    };

    \node[a, cell00] at (0,1) {
      {namespace X \{}\\
      ~~int i = 0;\\
      {\}}
    };
  },
  part2/.pic = {
    \draw[-latex] (0,0) -- ++(0,4) node[left]{\itshape Visibility};
    \node at (0,1) {$-$};
    \node at (0,3) {$-$};
    \node[left] at (0,1) {Internal};
    \node[left] at (0,3) {External};


    \draw (0,2) -- ++(6,0);
    \draw (3,0) -- ++(0,4);


    \node[a,cell10] at (0,3) {
      {\scriptsize namespace X \{}\\
      {~~\scriptsize \color{gray}// .hpp (decl)}\\
      {~~\color{orange} extern} int {\color{DarkGreen} i};\\
      {~~\scriptsize \color{gray}// .cpp (def)}\\
      {~~int {\color{DarkGreen} i} = 0};\\
      {\scriptsize \}}\\
    };

  }
}


\begin{tikzpicture}
  \pic {part1};
\end{tikzpicture}

\begin{tikzpicture}
  \tikzset{cell00/.style={opacity=0}}
  \pic {part1};
  \pic {part2};
\end{tikzpicture}


\begin{tikzpicture}
  \tikzset{cell00/.style={opacity=0}}
  \pic {part1};
  \pic {part2};

  \node[a] at (0,1) {
    namespace {\color{orange} \textvisiblespace~\{}\\
    ~~int i = 0;\\
    {\color{orange} \}}
  };
\end{tikzpicture}


\begin{tikzpicture}
  \tikzset{cell00/.style={opacity=0}}
  \tikzset{cell10/.style={opacity=0}}
  \pic {part1};

  \tikzset{cell10/.style={opacity=0}}
  \pic {part2};

  \tikzset{a/.append style={xshift=-5pt}}

  \node[a] at (0,3) {
    {\scriptsize namespace X \{}\\
    {~~\scriptsize \color{gray}// .hpp (decl)}\\
    {~~\color{orange} [extern]} void foo();\\
    {~~\scriptsize \color{gray}// .cpp (def)}\\
    {~~void foo() \{\ldots\}}\\
    {\scriptsize \}}\\
  };


  \node[a] at (0,1) {
    namespace {\color{orange} \textvisiblespace~\{}\\
    ~~void foo() \{\ldots\}\\
    {\color{orange} \}}
  };
\end{tikzpicture}


% Yellow:
\definecolor{BgYellow}{HTML}{FFF59C}
\definecolor{FrameYellow}{HTML}{F7A600}

\newtcolorbox{YStkyNote}[1][]{%
    enhanced,
    before skip=2mm,after skip=2mm, 
    width=0.4\textwidth, % width of the sticky note
    boxrule=0.2mm,
    colback=BgYellow, colframe=FrameYellow, % Colors
    attach boxed title to top left={xshift=0cm,yshift*=0mm-\tcboxedtitleheight},
    varwidth boxed title*=-3cm,
    % The titlebox:
    boxed title style={frame code={%
        \path[left color=FrameYellow,right color=FrameYellow,
        middle color=FrameYellow]
        ([xshift=-0mm]frame.north west) -- ([xshift=0mm]frame.north east)
        [rounded corners=0mm]-- ([xshift=0mm,yshift=0mm]frame.north east)
        -- (frame.south east) -- (frame.south west)
        -- ([xshift=0mm,yshift=0mm]frame.north west)
        [sharp corners]-- cycle;
        },interior engine=empty,
    },
    sharp corners,rounded corners=southeast,arc is angular,arc=3mm,
    % The "folded paper" in the bottom right corner:
    underlay={%
        \path[fill=BgYellow!80!black] ([yshift=3mm]interior.south east)--++(-0.4,-0.1)--++(0.1,-0.2);
        \path[draw=FrameYellow,shorten <=-0.05mm,shorten >=-0.05mm,color=FrameYellow] ([yshift=3mm]interior.south east)--++(-0.4,-0.1)--++(0.1,-0.2);
        },
    drop fuzzy shadow, % Shadow
    fonttitle=\bfseries, 
    title={#1}
}

\begin{tikzpicture}

  \node[align=left, font=\tt] (A) {
    \begin{YStkyNote}[A.cpp]
\begin{verbatim}
int global_A;

void foo() {
  bar();
};     
\end{verbatim}
    \end{YStkyNote}
  };


  \node[align=left, font=\tt] (B) [right=2cm of A] {
    \begin{YStkyNote}[B.cpp]
\begin{verbatim}
int global_B;

void bar() {
  global_A += 1;
};     
\end{verbatim}
    \end{YStkyNote}
  };

\draw[<->] (A) -- (B);
\end{tikzpicture}


\end{document}
